% !TeX root = ../main.tex
% -*- coding: utf-8 -*-

\chapter{常用包}
\label{chpt:method}

\section{The Tikz 绘图Package}
\label{sec:method:tikz}


The {\scshape pdf}\ package, where ``{\scshape pdf}'' is supposed to mean ``portable
graphics format'' (or ``pretty, good, functional'' if you
prefer\dots), is a package for creating graphics in an ``inline''
manner. It defines a number of \TeX\ commands that draw
graphics. For example, the code \verb|\tikz \draw (0pt,0pt) -- (20pt,6pt);|
yields the line \tikz \draw (0pt,0pt) -- (20pt,6pt); and the code \verb|\tikz \fill[orange] (1ex,1ex) circle (1ex);| yields \tikz
\fill[orange] (1ex,1ex) circle (1ex);.

\begin{figure}[h]
    \centering
    \input{./figure/pgf}
    \caption{\label{fig:exmaple1} 示例图1}
\end{figure}

In a sense, when you use {\scshape pdf}\ you ``program'' your graphics, just
as you ``program'' your document when you use \TeX.  You get all
the advantages of the ``\TeX-approach to typesetting'' for your
graphics: quick creation of simple graphics, precise positioning, the
use of macros, often superior typography. You also inherit all the
disadvantages: steep learning curve, no \textsc{wysiwyg}, small
changes require a long recompilation time, and the code does not
really ``show'' how things will look like.





\begin{figure}
    \centering
    \input{./figure/process}
    \caption{\label{fig:exmaple2} 示例流程图2}
\end{figure}


\section{代码块}
\label{sec:method:code}

python 代码可以直接使用\textbf{python}环境

\begin{python}[caption={斐波那契Python}]
def fibonacci(n):
    # Fibonacci number
    if n < 0:
        return False
    if n <= 1:
        return n
    return fibonacci(n-2) + fibonacci(n-1)
\end{python}

C/C++ 代码可以直接使用\textbf{cpp}环境

\begin{cpp}[caption={斐波那契C++}]
unsigned long Fibonacci(int n)
{
    // Fibonacci start from 0
    if (n <= 1) 
    {
        return n;
    }
    else 
    {
        return Fibonacci(n - 1) + Fibonacci(n - 2);
    }
}
\end{cpp}

其他代码,使用\textbf{lstlisting}指明 \textbf{language}即可,如matlab代码

\begin{lstlisting}[caption={Matlab代码},language=Matlab]
function a = factorial(n)
% return n!
    if n==0
        a=1;
    else
        a=n * factorial(n-1);
    end
\end{lstlisting}